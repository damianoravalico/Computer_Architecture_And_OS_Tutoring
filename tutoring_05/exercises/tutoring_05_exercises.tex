\documentclass{article}
\usepackage[utf8]{inputenc}
\usepackage{geometry}
\usepackage{fancyhdr}
\usepackage{amssymb}
\usepackage{listings}
\usepackage[T1]{fontenc}

\def\code#1{\texttt{#1}}

\pagestyle{fancy}
\fancyhf{}
\lhead{Tutorato 5}
\chead{Architetture degli Elaboratori e Sistemi Operativi}
\rhead{\today}
\cfoot{\thepage}

\begin{document}
\section*{C - Comunicazione tra processi}
\subsection*{Esercizio 1}
Si scriva un programma C in cui, dato un vettore di interi, il processo padre genera due processi figli; il primo calcolerà il valore massimo e lo stamperà su stdout, il secondo calcolerà il valore minimo e lo stamperà su stdout. Il processo padre invece attenderà che entrambi i processi figli abbiano concluso l'esecuzione e successivamente calcolerà il valore medio stampandolo su stdout.

\subsection*{Esercizio 2}
Si scriva un programma C che abbia due processi, parent e child. Il processo padre esegue per $N$ volte il seguente ciclo:
\begin{itemize}
    \item scrive una riga su stdout
    \item dorme un secondo
    \item comunica al processo figlio che può continuare
    \item si ferma fin quando il processo figlio non gli comunica di continuare
\end{itemize}
Il processo figlio esegue all'infinito il seguente ciclo:
\begin{itemize}
    \item attende che il processo padre gli comunichi di partire
    \item scrive una riga su stdout
    \item dorme un secondo
    \item comunica al processo padre che può continuare
\end{itemize}
Quando il processo padre ha scritto la $N$-esima riga invia un segnale di terminazione al processo figlio, aspetta che termini e poi termina anch'esso.
 
Alla fine dell'esecuzione, con ad esempio $N = 2$, si avrà il seguente output:\newline
\code{[parent] data=01\newline[child ] data=01\newline[parent] data=02\newline[child ] data=02\newline}

\subsection*{Esercizio 3}
Si scriva un programma C, utilizzando una o più pipe, in cui il processo padre origina due processi figli A e B. Il processo A «produce» una stringa RA e il processo B «produce» una stringa RB. Il processo padre aspetta la conclusione dei due processi figli; quando entrambi hanno concluso l’esecuzione, concatena RA e RB e scrive il risultato su stdout. 

\subsection*{Esercizio 4}
Si scriva un programma C in cui il processo padre crea $N$ processi figli; il processo figlio $i$-esimo calcola $fibonacci(i)$ e lo inserisce all'interno di una memoria condivisa. Il processo padre aspetta che tutti i figli abbiano terminato per poi scrivere su stdout i valori di fibonacci restituiti dai processi figli.

\end{document}
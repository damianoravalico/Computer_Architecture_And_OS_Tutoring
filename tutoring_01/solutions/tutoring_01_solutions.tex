\documentclass{article}
\usepackage[utf8]{inputenc}
\usepackage{geometry}
\usepackage{fancyhdr}
\usepackage{amsmath}

\newcommand*\xor{\oplus}
\def\code#1{\texttt{#1}}

\pagestyle{fancy}
\fancyhf{}
\lhead{Tutorato 1}
\chead{Architetture degli Elaboratori e Sistemi Operativi}
\rhead{\today}
\cfoot{\thepage}

\begin{document}

\section*{Binario, altre basi e rappresentazioni}
\subsection*{Esercizio 1}
\begin{itemize}
\item ${101110}_2 = 1\times2^5 + 0\times2^4 + 1\times2^3 + 1\times2^2 + 1\times2^1 + 0\times2^0 = 32 + 0 + 8 + 4 + 2 + 0 = 46_{10}$ 
\item ${100001}_2 = 1\times2^5 + 1\times2^0 = 33_{10}$
\item ${110111101}_2 = 1\times2^8 + 1\times2^7 + 1\times2^5 + 1\times2^4 + 1\times2^3 + 1\times2^2 + 1\times2^0 = 445_{10}$
\end{itemize}

\subsection*{Esercizio 2}
\begin{itemize}
\item ${1670}_8 = 1\times8^3 + 6\times8^2 + 7\times8^1 + 0\times8^0 =  512 + 384 + 56 = 952_{10}$
\item ${1043}_8 = 1\times8^3 + 4\times8^1 + 3\times8^0 = 547_{10}$
\item ${25012}_8 = 2\times8^4 + 5\times8^3 + 1\times8^1 + 2\times8^0 = 10762_{10}$
\end{itemize}

\subsection*{Esercizio 3}
\begin{itemize}
\item ${11F}_{16} = 1\times16^2 + 1\times16^1 + 15\times16^0 = 256 + 16 + 15 = 287_{10}$
\item ${4CD}_{16} = 4\times16^2 + 12\times16^1 + 13\times16^0 = 1229_{10}$
\item ${10043}_{16} = 1\times16^4 + 4\times16^1 + 3\times16^0 = 65603_{10}$
\end{itemize}

\subsection*{Esercizio 4}
\begin{itemize}
\item
\begin{align*}
10001&+  &17 +\\
 1110&=  &14 = \\
11111&  &31_{10}
\end{align*}
\item
\begin{align*}
100^1 1^1 1^1 0^1 1&+  &77\\
        111&=  &7\\
1010100&  &84_{10}
\end{align*}
\end{itemize}

\subsection*{Esercizio 5}
\begin{itemize}
\item ${110110}_{c2} = 1\times-2^5 + 1\times2^4 + 0\times2^3 + 1\times2^2 + 1\times2^1 + 0\times2^0 = -32 + 16 + 4 + 2 = -10_{10}$
\item ${01011}_{c2} = 0\times2^4 + 1\times2^3 + 1\times2^1 + 1\times2^0 = 11_{10}$
\item ${11110001}_{c2} = 1\times-2^7 + 1\times2^6 + 1\times2^5 + 1\times2^4 + 1\times2^0 = -15_{10}$
\end{itemize}

\subsection*{Esercizio 6}
\begin{itemize}
\item ${010101}_{e128} = 21 - 128 = -107$
\item ${1101}_{e16} = 13 - 16 = -3$
\item ${111100010}_{e64} = 482 - 64 = 418$
\end{itemize}

\section*{Circuiti logici e algebra di Boole}
\subsection*{Esercizio 7}
Si veda \code{logisim/exercise07\_solution.cisc}

\subsection*{Esercizio 8}
\begin{align*}
f &= \neg(x \lor (w \land z) \lor (x \land \neg y)) \quad &\text{raccoglimento}\\
&= \neg(x \land (1 \lor \neg y) \lor (w \land z)) \quad &1\lor\neg y = 1\\
&= \neg(x \lor (w \land z)) \quad &\text{De Morgan}\\
&= \neg x \land \neg(w \land z) = \neg(x \land w \land z) \quad &\text{De Morgan}\\
&= \neg x \land(\neg w \lor \neg z)
\end{align*}

\subsection*{Esercizio 9}
Si veda \code{logisim/exercise09\_solution.cisc}

\subsection*{Esercizio 10}
Si veda \code{logisim/exercise10\_solution.cisc}

\subsection*{Esercizio 11}
Si veda \code{logisim/exercise11\_solution.cisc}

\subsection*{Esercizio 12}
Si veda \code{logisim/exercise12\_solution.cisc}

\end{document}

\documentclass{article}
\usepackage[utf8]{inputenc}
\usepackage{geometry}
\usepackage{fancyhdr}
\usepackage{amssymb}
\usepackage{listings}
\usepackage{color}

\newcommand*\xor{\oplus}

\definecolor{dkgreen}{rgb}{0,0.6,0}
\definecolor{gray}{rgb}{0.5,0.5,0.5}
\definecolor{mauve}{rgb}{0.58,0,0.82}

\lstset{frame=tb,
  language=C,
  aboveskip=3mm,
  belowskip=3mm,
  showstringspaces=false,
  columns=flexible,
  basicstyle={\small\ttfamily},
  numbers=none,
  numberstyle=\tiny\color{gray},
  keywordstyle=\color{blue},
  commentstyle=\color{dkgreen},
  stringstyle=\color{mauve},
  breaklines=true,
  breakatwhitespace=true,
  tabsize=3
}

\pagestyle{fancy}
\fancyhf{}
\lhead{Tutorato 2}
\chead{Architetture degli Elaboratori e Sistemi Operativi}
\rhead{\today}
\cfoot{\thepage}

\begin{document}

\section*{Registri}
\subsection*{Esercizio 1}
Si realizzi un file register con 4 registri a 8 bit che abbia un unico input e un unico output. Il circuito deve avere un bit per \textit{write enable}, uno per \textit{read enable} e un indice per poter selezionare il registro che si vuole utilizzare.

\section*{Assembly}
\subsection*{Esercizio 2}
Si descriva, tramite linguaggio naturale, lo scopo del seguente frammento di codice assembly.
\begin{lstlisting}
		MOV		R1, #10
		MOV		R2, #0
loop
		CMP		R2, R1
		BEQ		end_loop
		ADD		R2, R2, #1
		ADD		R0, R0, R2
		B		loop
end_loop
\end{lstlisting}

\subsection*{Esercizio 3}
Si descriva, tramite linguaggio naturale, lo scopo del seguente frammento di codice assembly.
\begin{lstlisting}
		MOV		R1, #13
		MOV		R2, #4
		MOV		R3, #0
loop
		CMP		R3, R2
		BEQ		end_loop
		ADD		R0, R0, R1
		ADD		R3, R3, #1
		B		loop
end_loop
\end{lstlisting}

\subsection*{Esercizio 4}
Si traduca il seguente frammento di codice C nel linguaggio assembly. Il risultato deve essere salvato in \textit{R0}.
\begin{lstlisting}
int main(int argc, char** argv) {
	int firstNumber = 88;
	int secondNumber = 24;
	int result = 0;
	if (firstNumber >= secondNumber) {
		result = firstNumber + secondNumber;
	} else {
		result = secondNumber - firstNumber;
	}
}
\end{lstlisting}

\subsection*{Esercizio 5}
Si traduca il seguente frammento di codice C nel linguaggio assembly. Il risultato deve essere salvato in \textit{R0}.
\begin{lstlisting}
int main(int argc, char** argv) {
	int myArray[] = {21, 5, 66, 14, 37};
	int i = 0;
	int result = 0;
	while (i < 5) {
		if (myArray[i] >= 30) {
			result = result + myArray[i];
		}
	}
}
\end{lstlisting}

\subsection*{Esercizio 6}
Dato un vettore di lunghezza 4, i cui elementi appartengono a $\mathbb{Z}$, si trovi il valore massimo e lo si memorizzi nel registro \textit{R0}.

\subsection*{Esercizio 7}
Dato un numero naturale $n$, si sommino i primi $n \in \mathbb{N} \setminus \{0\}$ numeri tramite la formula di Gauss, senza utilizzare le istruzioni di moltiplicazione e divisione. Il risultato deve essere salvato in \textit{R0}.

\end{document}

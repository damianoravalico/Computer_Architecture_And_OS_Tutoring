\documentclass{article}
\usepackage[utf8]{inputenc}
\usepackage{geometry}
\usepackage{fancyhdr}
\usepackage{amssymb}
\usepackage{listings}
\usepackage[T1]{fontenc}

\def\code#1{\texttt{#1}}

\pagestyle{fancy}
\fancyhf{}
\lhead{Tutorato 7}
\chead{Architetture degli Elaboratori e Sistemi Operativi}
\rhead{\today}
\cfoot{\thepage}

\begin{document}
\section*{C - Comunicazione tra processi}
\subsection*{Esercizio 1}
Si scriva un programma C dove N processi vengono generati dal processo padre; essi condividono un segmento di memoria in cui, il processo $i$-esimo scriverà il carattere 'a' + $i$ in una parte di dimensione $\frac{N}{M}$, con offset pari a $i * \frac{N}{M}$, dove $N = 16 * M$ con $M = 6$. Ogni processo dovrà scrivere su stdout il carattere corrispondente e l'offset da cui ha iniziato a scriverlo. Il processo padre attenderà che tutti i figli abbiano concluso l'esecuzione per poi scrivere su stdout ogni carattere salvato nella memoria condivisa.

\subsection*{Esercizio 2}
Si scriva un programma C in cui due processi si scambiano segnali, con lo scopo di simulare una partita a tennistavolo. Il primo processo scriverà su stdout una stringa corrispondente al proprio colpo ("Ping" o "Pong"), successivamente invierà un segnale al secondo processo che risponderà, e così via fino a quando non saranno stati eseguiti 10 scambi. La partita inizia con uno dei due processi scelto casualmente (il primo colpo potrebbe quindi essere un "Pong").

\section*{Threads}
\subsection*{Esercizio 3}
Si scriva un programma C in cui viene implementata la funzione \code{long computeFunctioValue()}, in cui viene eseguito il seguente calcolo $y = x^4 - x^3 + x^2$. Tale calcolo viene  effettuato utilizzando 3 threads:
\begin{itemize}
\item un thread calcola $x^4$
\item un thread calcola $x^3$
\item un thread calcola $x^2$
\end{itemize}
Tale funzione crea i 3 thread, attende che i risultati siano pronti, calcola $y$ e lo stampa su stdout.

\subsection*{Esercizio 4}
Si scriva un programma C in cui il thread principale riempie un vettore di dimensione 9000000 con numeri random tra 0 e 9. Successivamente crea \code{N\_THREADS} threads che si coordineranno per calcolare la somma degli elementi del vettore. La variabile contenente tale valore deve essere condivisa tra tutti i thread, i quali lavoreranno direttamente su di essa. Ciascun thread calcola la somma relativa solamente ad una porzione del vettore, di dimensione uguale per tutti, con offset diverso.

\end{document}
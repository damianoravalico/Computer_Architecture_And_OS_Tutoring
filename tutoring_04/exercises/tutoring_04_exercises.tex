\documentclass{article}
\usepackage[utf8]{inputenc}
\usepackage{geometry}
\usepackage{fancyhdr}
\usepackage{amssymb}
\usepackage{listings}
\usepackage[T1]{fontenc}

\def\code#1{\texttt{#1}}

\pagestyle{fancy}
\fancyhf{}
\lhead{Tutorato 2}
\chead{Architetture degli Elaboratori e Sistemi Operativi}
\rhead{\today}
\cfoot{\thepage}

\begin{document}
\section*{Assembly}
\subsection*{Esercizio 1}
Si scriva un programma assembly che, tramite una subroutine, esegua la divisione intera tra due valori restituendo quoziente e resto, rispettivamente in $R0$ e $R1$.

\subsection*{Esercizio 2}
Si scriva un programma assembly che calcoli la funzione di fibonacci, il cui algoritmo non ricorsivo è implementato in una subroutine, restituendo $f(R0)$ in $R0$.

\section*{Shell Unix}
\subsection*{Esercizio 3}
Si descriva l'effetto dell'esecuzione sequenziale dei seguenti comandi, motivando la risposta.
\newline
\code{
> cd\\
> mkdir ex3\\
> chmod 444 ex3\\
> cd ex2
}

\subsection*{Esercizio 4}
Si eseguano le seguenti operazioni:
\begin{enumerate}
\item Creare una directory denominata \code{tut3}
\item Cambiare i permessi della suddetta directory in modo tale che il proprietario possa fare tutto, il gruppo possa leggere e scrivere mentre il resto del mondo possa solamente leggere
\item Creare un file all'interno della directory denominato \code{ex4.txt}
\item Scrivere nel file alcune righe di testo
\item Stampare il numero di caratteri presenti nel file
\item Eliminare l'intera cartella
\end{enumerate}

\subsection*{Esercizio 5}
Si eseguano le seguenti operazioni:
\begin{enumerate}
\item Creare un file denominato \code{ex5.txt}
\item I permessi di questo file dovranno essere \code{rwx} per il proprietario e nessuno per gruppo e resto del mondo
\item Riempire tale file con la lista dei filenames (compresi i files che iniziano con \code{.}) presenti in una directory a scelta.
\end{enumerate}

\subsection*{Esercizio 6}
\code{grep} è un comando filtro che ricerca nel file di testo specificato le linee che corrispondono ad un dato pattern (espressioni regolari o stringhe letterali), e produce un elenco delle corrispondenze.\\

Si salvi l'elenco di tutti i file con estensione \code{pdf} con i relativi permessi e ordinati rispetto alla data di modifica, selezionando solamente quelli la cui data di modifica risale ad mese di aprile in un file di testo denominato \code{ex6.txt}. Se una delle corrispondenze date da \code{grep} dovesse contenere una stringa "Apr" che non sia la data di ultima modifica, può essere comunque considerata valida.

\end{document}
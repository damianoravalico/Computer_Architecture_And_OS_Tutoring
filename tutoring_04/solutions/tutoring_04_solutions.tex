\documentclass{article}
\usepackage[utf8]{inputenc}
\usepackage{geometry}
\usepackage{fancyhdr}
\usepackage{amssymb}
\usepackage{listings}
\usepackage[T1]{fontenc}

\def\code#1{\texttt{#1}}
\newcommand\tab[1][0.4cm]{\hspace*{#1}}

\pagestyle{fancy}
\fancyhf{}
\lhead{Tutorato 4}
\chead{Architetture degli Elaboratori e Sistemi Operativi}
\rhead{\today}
\cfoot{\thepage}

\begin{document}

\section*{Assembly}
\subsection*{Esercizio 1}
Si veda \code{assembly/esercizio01\_solution.s}

\subsection*{Esercizio 2}
Si veda \code{assembly/esercizio02\_solution.s}

\section*{Shell Unix}
\subsection*{Esercizio 3}
Tramite questi comandi viene creata una nuova cartella i cui permessi, per ciascuna categoria, sono impostati a \code{4} che corrisponde a "sola lettura". Quando si cercherà quindi di accedervi tramite \code{cd}, non avendo i permessi, l'operazione avrà esito negativo.

\subsection*{Esercizio 4}
\code{> mkdir tut3\\
> chmod 764 tut3\\
> cd tut3\\
> touch ex4.txt\\
> vi ex4.txt\\
> queste sono\\
\tab alcune righe\\
\tab di prova\\
press esc\\
:wq\\
press enter\\
> wc -m ex4.txt\\
> cd ..\\
> rm -r tut3
}

\subsection*{Esercizio 5}
\code{> touch ex5.txt\\
> chmod 700 ex5.txt\\
> ls -a . > ex5.txt
}

\subsection*{Esercizio 6}
\code{> touch ./Documents/ex6.txt\\
> ls -lt *.pdf | grep Apr > ./Documents/ex6.txt}

\end{document}